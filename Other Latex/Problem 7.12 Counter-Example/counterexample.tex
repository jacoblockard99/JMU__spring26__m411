\documentclass{article}
\usepackage{amsmath} 
\usepackage{amsfonts}
\usepackage{amssymb}
\usepackage[parfill]{parskip}
\usepackage{esvect}
\usepackage[thinc]{esdiff}
\usepackage{mathtools}
\usepackage{physics}
\usepackage{amsthm}
\usepackage{xfrac}
\usepackage{enumitem}

\DeclareMathOperator{\lub}{lub}
\DeclareMathOperator{\gub}{gub}

\begin{document}

\textbf{Context from Theorem 7.1:} 
\begin{quote}
    \itshape
    Suppose that $\{a_k\}_{k=1}^\infty$ is a sequence of monotonically
    decreasing nonnegative numbers.
\end{quote}
\textbf{Problem 7.12(b):} 
\begin{quote}
    \itshape
    Let $s_n$ be the $n$-th partial sum of $\sum a_k$ and
    $t_m$ be the $m$-th partial sum of $\sum 2^{\ell} a_{2^\ell}$.
    Use the inequality above to prove that $s_{2^{m+1}-1} \leq t_m$
    for all $m \in \mathbb{N}$.
\end{quote}

\textbf{Counter-example: } Let $\{a_k\}_{k=1}^\infty$ be such that
\begin{align*}
    a_1 &= 3 \,, \\
    a_2 &= 2 \,, \\
    a_3 &= 1 \,,
\end{align*}
and $a_k = 0$ for all $k > 3$. Clearly, $\{a_k\}_{k=1}^\infty$ is
monotonically decreasing, and all its terms are nonnegative.
Let $m = 1 \in \mathbb{N}$. Then,
\begin{align*}
    s_{2^{m+1} - 1} = s_{2^2-1}
    = s_3
    = a_1 + a_2 + a_3
    = 3 + 2 + 1
    = 6 \,, \\
    t_m
    = t_1
    = 2^1a_{2^1}
    = 2a_2
    = 4 \,,
\end{align*}
so $s_{2^{m+1} - 1} > t_m$ in this case.
\qedsymbol

\textbf{Proposition: } Perhaps the statement should read:
\begin{quote}
    \itshape
    Let $s_n$ be the $n$-th partial sum of $\sum a_k$ and
    $t_m$ be the $m$-th partial sum of $\sum_{\ell=0}^\infty 2^{\ell} a_{2^\ell}$.
    Use the inequality above to prove that $s_{2^{m+1}-1} \leq t_m$
    for all $m \in \mathbb{N}$.
\end{quote}
Maybe this was the original intention, but it seems odd for the
one series to start at $1$ and the other to start at $0$.

\end{document}