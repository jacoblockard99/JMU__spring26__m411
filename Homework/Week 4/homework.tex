\documentclass{article}
\usepackage{amsmath} 
\usepackage{amsfonts}
\usepackage{amssymb}
\usepackage[parfill]{parskip}
\usepackage{esvect}
\usepackage[thinc]{esdiff}
\usepackage{mathtools}
\usepackage{physics}
\usepackage{amsthm}
\usepackage{xfrac}
\usepackage{enumitem}
\usepackage{nccmath}

\DeclareMathOperator{\lub}{lub}
\DeclareMathOperator{\gub}{gub}

\begin{document}

\title{MATH 411: Week 4}
\author{Jacob Lockard}
\date{19 February 2026}

\maketitle

\section*{Problem 7.32}

\begin{quote}
    \itshape
    Let $c_n = \sum_{k=0}^n c_n$ (where $c_n$ is defined as above). Prove
    that $C_n = a_0 B_{n-1} + \dots + a_nB_0$ where
    $B_n = \sum_{k=0}^n b_n$. Define $\beta_n = B_n - B$ and
    $A_n = \sum_{k=0}^n a_n$. Explain why the following are true:
    \begin{enumerate}[label=(\alph*)]
        \item $\lim_{n\to\infty} \beta_n = 0$
        \item For all $N \geq 0$, $C_n = A_nB + a_0\beta_n + a_1\beta_{n=1} + \dots + a_n\beta_0$
        \item $\lim_{n\to\infty} A_n B = AB$
    \end{enumerate}
\end{quote}

Let $\sum_{n=0}^\infty A_n$ be an absolutely convergent series with
$\sum_{n=0}^\infty a_n = A$, and let $\sum_{n=0}^\infty b_n$ be a convergent
series with $\sum_{n=0}^\infty b_n = B$.
For all $n \geq 0$, define $c_n = \sum_{k=0}^n a_k b_{n-k}$, $C_n = \sum_{k=0}^n c_k$,
and $\beta_n = B_n - B$.

Let $n \geq 0$.
We'll show that $C_n = a_0B_{n} + a_1B_{n-1} + \dots + a_nB_0$ where
$B_n = \sum_{k=0}^n b_k$.
We have:
\begin{align*}
    C_n &= \sum_{k=0}^n c_k \\
    &= (a_0b_0) + (a_0b_1 + a_1b_0) + \dots + (a_0b_n + a_1 b_{n-1} + \dots + a_n b_0) \\
    &= a_0 (b_0 + b_1 + \dots + b_n) + a_1 (b_0 + b_1 + \dots b_{n-1}) + \dots + a_n b_0 \\
    &= a_0B_n + a_1B_{n-1} + \dots + a_nB_0 \,.
\end{align*}

\textbf{(a)} We'll show that
\begin{align*}
    \lim_{n\to\infty} \beta_n = 0 \,.
\end{align*}
Since $\sum b_n = B$, we know $\{B_n\} \to B$, and thus
$\{B_n - B\} \to B-B = 0$.

\textbf{(b)} Let $n \geq 0$. We'll show that
\begin{align*}
    C_n = A_nB + a_0\beta_n + a_1\beta_{n-1} + \dots + a_n\beta_0 \,.
\end{align*}
We have:
\begin{align*}
    \sum_{k=0}^n a_k\beta_{n-k} &= \sum_{k=0}^n a_k (B_{n-k} - B) \\
    &= \sum_{k=0}^k a_k B_{n-k} - \sum_{k=0}^n a_k B \\
    &= \sum_{k=0}^k a_k B_{n-k} - A_n B \,.
\end{align*}
We conclude:
\begin{align*}
    C_n = \sum_{k_0}^k a_k B_{n-k} = \sum_{k=0}^n a_k \beta_{n-k} + A_nB \,.
\end{align*}

\textbf{(c)} We will show that
\begin{align*}
    \lim_{n \to \infty} A_nB = AB \,.
\end{align*}
Since $\{A_n\} \to A$ and $\{B\}_{n=0}^\infty \to B$, Problem 5.13
ensures $\{A_n B\} \to AB$.

\qedsymbol

\section*{Problem 7.33}

\begin{quote}
    \itshape
    Let $\gamma = \sum_{n=0}^\infty |a_n|$. Use the first result in problem 7.32 to
    prove that for any $\delta_1 > 0$, there exists an $N_1$ such that for all
    $n > N_1$,
    \begin{align*}
        |\alpha_0 \beta_n + \alpha_1 \beta_{n-1} + \dots + \alpha_{n-N_1} \beta_{N_1}| \leq \delta_1 \gamma \,.
    \end{align*}
\end{quote}

Let $\gamma = \sum_{n=0}^\infty |a_n|$ and $\delta_1 > 0$.
We'll show that there exists an $N_1 \in \mathbb{N}$ such that for all
$n > N_1$,
\begin{align*}
    |\alpha_0 \beta_n + \alpha_1 \beta_{n-1} + \dots + \alpha_{n-N_1} \beta_{N_1}| \leq \delta_1 \gamma \,.
\end{align*}
Problem 7.32 ensures that $\{\beta_n\} \to 0$, so there's an $N_1 \in \mathbb{N}$
such that if $n \geq N_1$, then
\begin{align*}
    |\beta_n| <  \delta_1 \,.
\end{align*}
Since $\gamma = \sum_{n=0}^\infty |a_n|$ is increasing and $\delta_1 > 0$, we know
\begin{align*}
    \sum_{k=0}^{n-N_1} |a_k| &\leq \gamma \\
    \sum_{k=0}^{n-N_1} \delta_1|a_k| &\leq \delta_1 \gamma \,,
\end{align*}
for all $n \geq N_1$.
If $0 \leq k \leq n - N_1$, then $n - k \geq N_1$, so
\begin{align*}
    |\beta_{n-k}| < \delta_1 \,.
\end{align*}
Thus,
\begin{align*}
    \sum_{k=0}^{n-N_1} |\beta_{n-k}| |a_k| < \delta_1 \gamma \,.
\end{align*}
The triangle inequality now ensures
\begin{align*}
    \Biggl| \sum_{k=0}^{n-N_1} \beta_{n-k} a_k \Biggr| \leq \sum_{k=0}^{n-N_1} |\beta_{n-k} a_k| < \delta_1 \gamma \,.
\end{align*}

\qedsymbol

\section*{Problem 7.34}

\begin{quote}
    \itshape
    Following along with the notation in the previous two problems:
    \begin{enumerate}[label=(\alph*)]
        \item Explain why we can choose an $M$ with $M > |\beta_n|$ for all $n$.
        \item Then use the fact that $\sum_{n=0}^\infty |a_n|$ converges to show
        that for any $\delta_2 > 0$, there exists an $N_2$ such that
        for any $n > N_2$ we have
        \begin{align*}
            |a_{N_2}\beta_{n-N_2} + a_{N_2+1}\beta_{n-N_2-1} + \dots + a_n\beta_0| < M \delta_2 \,.
        \end{align*}
    \end{enumerate}
\end{quote}

\textbf{(a)} Since $\{\beta_n\}$ converges, it must be bounded.

\textbf{(b)} Let $\delta_2 > 0$.
Since $\sum_{n=0}^\infty |a_n|$ converges, the Cuachy criteron
ensures
\begin{align*}
    \sum_{k=N_2}^{n} |a_k| &\leq \delta_2 \\
    \sum_{k=0}^{n-N_2} |a_{N_2 + k}| &\leq \delta_2 \\
    \sum_{k=0}^{n-N_2} |a_{N_2 + k}| M &\leq M\delta_2 \,.
\end{align*}
By construction, $M > |\beta_{n-N_2-k}|$ for all $k \in \{0, 1, \dots, n - N_2\}$.
So we conclude:
\begin{align*}
    \sum_{k=0}^{n-N_2} |a_{N_2 + k}| |\beta_{n-N_2-k}| &< M\delta_2 \,.
\end{align*}
The triangle inequality then ensures
\begin{align*}
    \Biggl| \sum_{k=0}^{n-N_2} a_{N_2+k}\beta_{n-N_2-k} \Biggr| \leq \sum_{k=0}^{n-N_2} |a_{N_2 + k} \beta_{n-N_2-k}| < M\delta_2 \,.
\end{align*}

\qedsymbol

\section*{Problem 7.35}

\begin{quote}
    \itshape
    Now we use the work of the previous problems to finish the proof of Theorem 7.4.
    Choose appropriate values for $\delta_1$ and $\delta_2$ (as used in the previous
    two problems) to prove that for any $\epsilon > 0$ there exists an $N$
    (related to $N_1$ and $N_2$) such that for any $n \geq N$ we have
    \begin{align*}
        |a_0\beta_n + a_1\beta_{n-1}+\dots+a_n\beta_0| \leq \epsilon \,.
    \end{align*}
    Explain (using problem 7.32) why this means $\lim_{n\to\infty} C_n = AB$
    and thus $\sum_{n=0}^\infty c_n = AB$.
\end{quote}

We will show that $\sum_{n=0}^\infty c_n = AB$.
Let $\epsilon > 0$. Let $\delta_1 = \frac{\epsilon}{2\gamma}$, and
$\delta_2 = \frac{\epsilon}{2M}$.
Let $N = \max\{N_1, N_2\}$ and $n \geq N$. We have:
\begin{align*}
    \Biggl| \sum_{k=0}^n a_k\beta_{n-k} \Biggr|
    &= 
    \Biggl| \sum_{k=0}^{n-N} a_k\beta_{n-k}
    + \sum_{k={n-M+1}}^n a_k\beta_{n-k} \Biggr| \\
    &\leq \
    \Biggl| \sum_{k=0}^{n-N} a_k\beta_{n-k} \Biggr|
    + \Biggl| \sum_{k={n-M+1}}^n a_k\beta_{n-k} \Biggr| \\
    &< \delta_1 \gamma + M\delta_2 \\
    &= \frac{\epsilon}{2} + \frac{\epsilon}{2} \\
    &= \epsilon \,.
\end{align*}
So $\lim_{n\to\infty} \sum_{k=0}^n a_k\beta_{n-k} = 0$.
Then we have:
\begin{align*}
    \lim_{n\to\infty} C_n &= \lim_{n\to\infty} (A_nB) + \lim_{n\to\infty} \sum_{k=0}^n a_k \beta_{n-k} \\
    &= AB + 0 = AB \,,
\end{align*}
by our results in Problem 7.32.

\qedsymbol

\section*{Problem 7.36}

\begin{quote}
    \itshape
    Suppose that $\sum_{n=0}^\infty a_n$ and $\sum_{n=0}^\infty b_n$ are
    both absolutely convergent series. Define $c_n = \sum_{k=0}^n a_kb_{n-k}$.
    Use Theorem 7.4 to prove that $\sum_{n=0}^\infty c_n$ is absolutely
    convergent as well.
\end{quote}

Let $\sum_{n=0}^\infty a_n$ and $\sum_{n=0}^\infty b_n$ converge absolutely.
Define $c_n = \sum_{k=0}^n a_k b_{n-k}$, and $\overline{c_n} = \sum_{k=0}^n |a_k| |b_{n-k}|$.
We'll show that
$\sum_{n=0}^\infty c_n$ converges absolutely.
We notice that
\begin{align*}
    |c_n| = \Biggl| \sum_{k=0}^n a_k b_{n-k} \Biggr|
    \leq \sum_{k=0}^n |a_k b_{n-k}|
    = \sum_{k=0}^n |a_k| |b_{n-k}|
    = \overline{c_n} \,.
\end{align*}
Since $\sum_{n=0}^\infty |a_n|$ and $\sum_{n=0}^\infty |b_n|$ both
converge absolutely, Theorem 7.4 ensures that $\sum_{n=0}^\infty \overline{c_n}$
converges. The comparison test then ensures $\sum_{n=0}^\infty |c_n|$
converges. So $\sum_{n=0}^\infty c_n$ converges absolutely.

\qedsymbol

\section*{Problem 7.37}

\begin{quote}
    \itshape
    Let $\sum a_n$ be a divergent series where $a_n > 0$ for all $n$. Prove
    that $\sum \frac{a_n}{1+a_n}$ diverges.
\end{quote}

Let $\sum a_n$ be a divergent series with $a_n > 0$ for all $n$.
We'll show that $\sum \frac{a_n}{1+a_n} = \sum \frac{1}{\sfrac{1}{a_n} + 1}$ diverges.

Assume $\limsup_{n\to\infty} a_n = \infty$.
Then there's a subsequence $\{a_n'\}$ of $\{a_n\}$ with $\{a_n'\} \to \infty$.
So $\{\frac{1}{a_n'} \} \to 0$, $\{ \frac{1}{a_n'} + 1 \} \to 1$, and $\frac{1}{\sfrac{1}{a_n'} + 1} \to 1$.
Since there's a subsequence $\{\frac{1}{\sfrac{1}{a_n'} + 1}\}$ of $\{\frac{a_n}{1+a_n}\}$
that doesn't converge to 0, we know $\{\frac{a_n}{1+a_n} \} \not\to 0$.

Assume $\limsup_{n\to\infty} = L > 0$.
Then there's some subsequence $\{a_n'\}$ of $\{a_n\}$ with
$\{a_n'\} \to L$. Since $\{a_n' + 1\} \to L + 1$ and $1 + a_n'$ is never zero,
we have
\begin{align*}
    \Bigl\{ \frac{a_n'}{1+a_n'} \Bigr\} \to \frac{L}{L+1} \neq 0 \,.
\end{align*}
Since there's a subsequence $\Bigl\{ \frac{a_n'}{1+a_n'} \Bigr\}$ of
$\Bigl\{ \frac{a_n}{1+a_n} \Bigr\}$ that doesn't converge to 0,
we know $\Bigl\{ \frac{a_n}{1+a_n} \Bigr\} \not\to 0$.

Assume $\limsup_{n\to\infty} a_n = 0$. Then there's some $N \in \mathbb{N}$
such that if $n \geq N$, then $a_n < 1$, or $1 - a_n > 0$.
So, if $n \geq N$, we have
\begin{align*}
    \frac{a_n}{1+a_n} > \frac{a_n}{1+a_n + (1 - a_n)} = \frac{a_n}{2} \,.
\end{align*}
Since $\sum a_n$ diverges, $\sum \frac{a_n}{2}$ also diverges, and the
comparison test ensures $\sum \frac{a_n}{1+a_n}$ diverges.

\qedsymbol

\end{document}