\documentclass{article}
\usepackage{amsmath} 
\usepackage{amsfonts}
\usepackage{amssymb}
\usepackage[parfill]{parskip}
\usepackage{esvect}
\usepackage[thinc]{esdiff}
\usepackage{mathtools}
\usepackage{physics}
\usepackage{amsthm}
\usepackage{xfrac}
\usepackage{enumitem}
\usepackage{nccmath}

\DeclareMathOperator{\lub}{lub}
\DeclareMathOperator{\gub}{gub}

\begin{document}

\title{MATH 411: Week 4}
\author{Jacob Lockard}
\date{13 February 2026}

\maketitle

\section*{Problem 7.31}

\begin{quote}
    \itshape
    Consider the series
    \begin{align*}
        \sum_{n=0}^\infty \frac{(-1)^n}{\sqrt{n+1}} \,.
    \end{align*}
    \begin{enumerate}[label=(\alph*)]
        \item Explain (briefly) why the series converges. (Does it converge
        absolutely?)
        \item If we take the product of the series with itself, then we have
        \begin{align*}
            c_n = (-1)^n \sum_{k=0}^\infty \frac{1}{\sqrt{(n-k+1)(k+1)}}
        \end{align*}
        (you should double check this). Show that each term in this sum is
        greater than or equal to $\frac{1}{\sqrt{n+1}}$, and thus $|c_n| \geq \sqrt{n+1}$.
        \item Explain (briefly) why the previous part implies $\sum c_n$ diverges.
    \end{enumerate}
\end{quote}

Define the series $\sum_{n=0}^\infty a_n$ by
\begin{align*}
    a_n = \frac{(-1)^n}{\sqrt{n+1}} \,.
\end{align*}

\textbf{(a)} We'll show that $\sum a_n$ converges absolutely.
We note that for any $n \in \mathbb{N}_0$,
\begin{align*}
    \lim_{n\to\infty} \frac{\mfrac{1}{\sqrt{n+1}}}{\mfrac{1}{\sqrt{n}}}
    = \lim_{n\to\infty} \sqrt{\frac{n+1}{n}}
    = \lim_{n\to\infty} \sqrt{1 + \frac{1}{n}}
    = 1\,.
\end{align*}
Since the sequence $\Bigl\{\mfrac{1}{n^{\sfrac{1}{2}}} \Bigr\}$
is a convergent $p$-series, the limit comparison test ensures
$\Bigl\{ \mfrac{1}{\sqrt{n+1}} \Bigr\}$ also converges. Since
\begin{align*}
    \Biggl| \frac{(-1)^n}{\sqrt{n+1}} \Biggr| = \frac{1}{\sqrt{n+1}} \,,
\end{align*}
we conclude that $\sum a_n$ converges absolutely.

\textbf{(b)} We will show that we have $|c_n| > \sqrt{n+1}$
for all $n$, where $\sum_{n=0}^\infty c_n$ is the Cauchy product of $\sum a_n$ with
itself. By definition,
\begin{align*}
    c_n = (-1)^n \sum_{k=0}^n \frac{1}{\sqrt{(n-k+1)(k+1)}} \,.
\end{align*}
If $n \in \mathbb{N}_0$ and $0 \leq k \leq n$, we have
\begin{align*}
    k &\leq n \\
    k\cdot k &\leq nk \\
    -k^2 &\geq -nk \\
    nk-k^2 + n + 1 &\geq n+1 \\
    (n-k+1)(k+1) &\geq n+1 \\
    \sqrt{(n-k+1)(k+1)} &\geq \sqrt{n+1} \\
    \frac{1}{\sqrt{(n-k+1)(k+1)}} &\geq \frac{1}{\sqrt{n+1}} \,,
\end{align*}
where the last line is justified by the fact that
the square root function is nonnegative and that here its arguments are nonzero.
Now we can say that for any $n \in \mathbb{N}_0$,
\begin{align*}
    \sum_{k=0}^n \frac{1}{\sqrt{(n-k+1)(k+1)}} &\geq \sum_{k=0}^n \frac{1}{\sqrt{n+1}} \\
    &\geq \frac{n+1}{\sqrt{n+1}} \\
    &= \sqrt{n+1} \,.
\end{align*}
So for any $n \in \mathbb{N}_0$, we have $|c_n| \geq \sqrt{n+1}$.

\textbf{(c)} We'll show that $\sum c_n$ diverges.
The series $\sum_{n=0}^\infty \sqrt{n+1}$ diverges, since its terms are unbounded.
The comparison test then ensures $\sum c_n$ also diverges.

\qedsymbol

\section*{Problem 7.32}

\begin{quote}
    \itshape
    Let $c_n = \sum_{k=0}^n c_n$ (where $c_n$ is defined as above). Prove
    that $C_n = a_0 B_{n-1} + \dots + a_nB_0$ where
    $B_n = \sum_{k=0}^n b_n$. Define $\beta_n = B_n - B$ and
    $A_n = \sum_{k=0}^n a_n$. Explain why the following are true:
    \begin{enumerate}[label=(\alph*)]
        \item $\lim_{n\to\infty} \beta_n = 0$
        \item For all $N \geq 0$, $C_n = A_nB + a_0\beta_n + a_1\beta_{n=1} + \dots + a_n\beta_0$
        \item $\lim_{n\to\infty} A_n B = AB$
    \end{enumerate}
\end{quote}

Let $\sum_{n=0}^\infty A_n$ be an absolutely convergent series with
$\sum_{n=0}^\infty a_n = A$, and let $\sum_{n=0}^\infty b_n$ be a convergent
series with $\sum_{n=0}^\infty b_n = B$.
For all $n \geq 0$, define $c_n = \sum_{k=0}^n a_k b_{n-k}$, $C_n = \sum_{k=0}^n c_k$,
and $\beta_n = B_n - B$.

Let $n \geq 0$.
We'll show that $C_n = a_0B_{n} + a_1B_{n-1} + \dots + a_nB_0$ where
$B_n = \sum_{k=0}^n b_k$.
We have:
\begin{align*}
    C_n &= \sum_{k=0}^n c_k \\
    &= (a_0b_0) + (a_0b_1 + a_1b_0) + \dots + (a_0b_n + a_1 b_{n-1} + \dots + a_n b_0) \\
    &= a_0 (b_0 + b_1 + \dots + b_n) + a_1 (b_0 + b_1 + \dots b_{n-1}) + \dots + a_n b_0 \\
    &= a_0B_n + a_1B_{n-1} + \dots + a_nB_0 \,.
\end{align*}

\textbf{(a)} We'll show that
\begin{align*}
    \lim_{n\to\infty} \beta_n = 0 \,.
\end{align*}
Since $\sum b_n = B$, we know $\{B_n\} \to B$, and thus
$\{B_n - B\} \to B-B = 0$.

\textbf{(b)} Let $n \geq 0$. We'll show that
\begin{align*}
    C_n = A_nB + a_0\beta_n + a_1\beta_{n-1} + \dots + a_n\beta_0 \,.
\end{align*}
We have:
\begin{align*}
    \sum_{k=0}^n a_k\beta_{n-k} &= \sum_{k=0}^n a_k (B_{n-k} - B) \\
    &= \sum_{k=0}^k a_k B_{n-k} - \sum_{k=0}^n a_k B \\
    &= \sum_{k=0}^k a_k B_{n-k} - A_n B \,.
\end{align*}
We conclude:
\begin{align*}
    C_n = \sum_{k_0}^k a_k B_{n-k} = \sum_{k=0}^n a_k \beta_{n-k} + A_nB \,.
\end{align*}

\textbf{(c)} We will show that
\begin{align*}
    \lim_{n \to \infty} A_nB = AB \,.
\end{align*}
Since $\{A_n\} \to A$ and $\{B\}_{n=0}^\infty \to B$, Problem 5.13
ensures $\{A_n B\} \to AB$.

\end{document}