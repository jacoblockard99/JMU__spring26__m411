\documentclass{article}
\usepackage{amsmath} 
\usepackage{amsfonts}
\usepackage{amssymb}
\usepackage[parfill]{parskip}
\usepackage{esvect}
\usepackage[thinc]{esdiff}
\usepackage{mathtools}
\usepackage{physics}
\usepackage{amsthm}
\usepackage{xfrac}

\DeclareMathOperator{\lub}{lub}
\DeclareMathOperator{\gub}{gub}

\begin{document}

\title{MATH 411: Week 1}
\author{Jacob Lockard}
\date{23 January 2026}

\maketitle

\section*{Definition}

Let $A \subseteq [-\infty, \infty]$. If $\infty \in A$, we define
$\lub A = \infty$. If $A = \{-\infty\}$, we define $\lub A = -\infty$.
Otherwise, we define $\lub A = \lub \bigl(A \setminus \{-\infty\}\bigr)$.

\section*{Problem 7.2}

\begin{quote}
    \itshape
    Let $\{a_n\}$ be a sequence of real numbers and let $E$ be its set of
    subsequential limits. Prove that $\limsup_{n \to \infty} a_n \in E$.
\end{quote}

Let $\{a_n\}$ be a sequence of real numbers and let $E \subseteq [-\infty, \infty]$ be its set of
subsequential limits. We will show that
$\limsup_{n \to \infty} a_n = \lub E \in E$.

Our definition ensures that if $\infty \notin E$, then $\lub E = -\infty$
or $\lub E \in \mathbb{R}$. So if $\lub E = \infty$, then $\infty \in E$.
If $E = \{-\infty\}$, our definition ensures $\lub E = -\infty \in E$.

If $\infty \notin E$ and $E \neq \{-\infty\}$, our definition
ensures that $\lub E = \lub A$, where $A$ is the set of accumulation points of
$\{a_n\}$. Since $\infty \notin E$, there is no subsequence of $\{a_n\}$ that
converges to $\infty$, so $\{a_n\}$ and thus $A$ is bounded above.
Problem 5.18 ensures $A$ is closed. Problem 4.9 then ensures
$\lub E = \lub A \in A \subseteq E$.
\qedsymbol

\section*{Problem 7.3}

\begin{quote}
    \itshape
    Let $\{a_n\}$ be a sequence of real numbers. If $\limsup_{n \to \infty} \in \mathbb{R}$,
    and $x > \limsup_{n \to \infty} a_n \in \mathbb{R}$, prove that there exists
    a natural number $N$ such that for all $n \geq N$ we have $a_n < x$.
\end{quote}

Let $\{a_n\}$ be a sequence of real numbers with $\alpha = \limsup_{n \to \infty} \in \mathbb{R}$,
and let $x > \alpha \in \mathbb{R}$.
We will show that there exists a natural number $N$ such that for all $n \geq N$ we have $a_n < x$.
Note that since $\limsup_{n\to\infty} \in \mathbb{R}$, there is no subsequence
of $\{a_n\}$ that converges to $\infty$, so $\{a_n\}$ is bounded.

Since $\alpha \in \mathbb{R}$, it is the least upper bound of the set of
accumulation points of $\{a_n\}$.
Since $\{a_n\}$ is bounded above, there exists some $M \in \mathbb{R}$ such that
$a_n \leq M$ for all $n$.
If $M < x$, then $a_n < x$ for all $n \geq 1$ and we're done.
So we assume $M \geq x$.

Problem 5.17 ensures that if there exists a subsequence of $\{a_n\}$ with terms
in $[x, M]$, then that subsequence---and thus $\{a_n\}$ itself---has an accumulation
point in $[x, M]$. But since $\alpha$ is an upper bound on the set of accumulation
points of $\{a_n\}$ and $x > \alpha$, we know that $[x, M]$ has no accumulation
points of $\{a_n\}$. We conclude that there is no subsequence of $\{a_n\}$ with
terms from $[x, M]$.
Further, since by definition there are no terms of $\{a_n\}$ greater than
$M$, we know that there is no subsequence of $\{a_n\}$ with terms from
$[x, \infty)$. So there are only finitely many
terms of $\{a_n\}$ greater than or equal to $x$.
\qedsymbol

\section*{Problem 7.5}

\begin{quote}
    \itshape
    If $\sum a_k$ converges then $\lim_{k \to \infty} a_k = 0$.
\end{quote}

Let $\sum a_k$ be a convergent series. We will show that $\lim_{k \to \infty} a_k = 0$.

Let $\epsilon > 0$. Problem 7.4 ensures that there is a natural $N$ such
that for all $m \geq n > N$, we have
\begin{align*}
    \Biggl| \sum_{k=n}^m a_k \Biggr| < \epsilon \,.
\end{align*}
If $m = n$, then
\begin{align*}
    \Biggl| \sum_{k=n}^n a_k \Biggr| < \epsilon &\\
    |a_n| < \epsilon &\\
    -\epsilon < a_n < \epsilon &\\
    0-\epsilon < a_n < 0+\epsilon &\\
    a_n \in (0-\epsilon, 0+\epsilon) &\,.
\end{align*}
So we've shown that for all $\epsilon > 0$, there's a natural number $N + 1$
such that if $n \geq N+1 > N$, we have $a_n \in (0-\epsilon, 0+\epsilon)$.
We conclude that $\lim_{k\to\infty} a_k = 0$.
\qedsymbol

\section*{Problem 7.6}

\begin{quote}
    \itshape
    If the series $\sum |a_k|$ converges then the series $\sum a_k$ converges as well.
\end{quote}

Let $\sum a_k$ be a series such that $\sum |a_k|$ converges. We will show that
$\sum a_k$ also converges.

Since $\sum |a_k|$ converges, Problem 7.4 ensures that there is a natural $N$
such that for all $m \geq n > N$, we have
\begin{align*}
    \Biggl| \sum_{k=n}^m |a_k| \Biggr| &< \epsilon
\end{align*}
Applying the triangle inequality, we get:
\begin{align*}
    \epsilon > \Biggl| \sum_{k=n}^m |a_k| \Biggr| \geq \Biggl|\Biggl| \sum_{k=n}^m a_k \Biggr|\Biggr| = \Biggl| \sum_{k=n}^m a_k \Biggr| \,.
\end{align*}
Problem 7.4 then ensures that $\sum a_k$ converges.
\qedsymbol

\section*{Problem 7.7}

\begin{quote}
    \itshape
    If $|a_k| \leq c_k$ for all $k \geq N_0$ where $N_0$ is a fixed integer, and
    $\sum c_k$ converges, then $\sum a_k$ converges as well.
\end{quote}

Let $N_0 \in \mathbb{N}$.
Let $\sum a_k$ be a series and $\sum c_k$ be a convergent series,
with $|a_k| \leq c_k$ for all $k \geq N_0$. We will show that $\sum a_k$
is also convergent.

Let $\epsilon > 0$. Since $\sum c_k$ converges, Problem 7.4 ensures that there
exists an $N \in \mathbb{N}$ such that for all $m \geq n > N$, we have
\begin{align*}
    \Biggl| \sum_{k=n}^m c_k \Biggr| < \epsilon \,.
\end{align*}
Let $M = \max \{N_0, N\}$. Then if $m \geq n > M$, since $0 \leq |a_k| \leq c_k$
for all $k \geq N_0$, we have:
\begin{align*}
    \sum_{k=n}^m |a_k| &\leq \sum_{k=n}^m c_k \\
    \Biggl| \sum_{k=n}^m |a_k| \Biggr| &\leq \Biggl| \sum_{k=n}^m c_k \Biggr| < \epsilon \,.
\end{align*}
Since for every $\epsilon > 0$ there exists an $M \in \mathbb{N}$ such that
for every $m \geq n > M$ we have $\Bigl|\sum_{k=n}^m |a_k|\Bigr| < \epsilon$,
Problem 7.4 ensures $\sum |a_k|$ converges. Problem 7.6 then ensures
$\sum a_k$ converges.
\qedsymbol

\section*{Problem 7.8}

\begin{quote}
    \itshape
    If $a_k \geq d_k \geq 0$ for all $k \geq N_0$ where $N_0$ is a fixed integer, and
    $\sum d_k$ diverges, then $\sum a_k$ diverges as well.
\end{quote}

Let $N_0 \in \mathbb{N}$. Let $\sum a_k$ be a series
and $\sum d_k$ be a divergent series, with $a_k \geq d_k \geq 0$ for all
$k \geq N_0$. We will show that $\sum a_k$ is also divergent.

Since $d_k \geq 0$, we know that $d_k = |d_k|$ and so
$|d_k| \leq a_k$ for all $k \geq N_0$.
If $\sum a_k$ converges, then Problem 7.7 ensures $\sum d_k$ converges.
Since $\sum d_k$ does not converge, we conclude that $\sum a_k$
does not converge.
\qedsymbol

\section*{Problem 7.9}

\begin{quote}
    \itshape
    Suppose that $a_k \geq 0$ for all $k$. Then the series $\sum a_k$
    converges if and only if the seqeunce of partial sums is bounded.
\end{quote}

Let $\sum a_k$ be a series with $a_k \geq 0$ for all $k$.
We will show that $\sum a_k$ converges if and only if its seqeunce of
partial sums $\{s_k\}$ is bounded.

Let $n \in \mathbb{N}$. Then:
\begin{align*}
    s_{n+1} = \sum_{k=1}^{n+1} a_k = \sum_{k=1}^n a_k + a_{n+1} = s_n + a_{n+1} \geq s_n \,,
\end{align*}
since $a_{n+1} \geq 0$. So $\{s_k\}$ is monotonically increasing.
Problem 5.23 then ensures $\{s_k\}$ converges if and only if it is bounded.
\qedsymbol

\section*{Problem 7.10}

\begin{quote}
    \itshape
    Let $r \in \mathbb{R}$ be fixed. Use mathematical induction to prove that for
    all $n \in \mathbb{N}$ we have
    \begin{align*}
        (1-r)(1+r+r^2+\dots+r^n) = 1 - r^{n+1} \,.
    \end{align*}
\end{quote}

Let $r \in \mathbb{R}$. For all $n \in \mathbb{N}$, let $P_n$ be
the statement that
\begin{align*}
    (1-r)(1+r+r^2+\dots+r^n) = 1 - r^{n+1} \,.
\end{align*}
We will show by induction that $P_n$ holds for all natural $n$.
$P_1$ holds:
\begin{align*}
    (1-r)(1+r^1) = (1-r)(1+r) = 1 - r^2 = 1 - r^{1+1} \,.
\end{align*}
Now assume that $P_n$ holds for some $n \in \mathbb{N}$, such that
\begin{align*}
    (1-r)(1+r+r^2+\dots+r^n) = 1 - r^{n+1} \,.
\end{align*}
Then:
\begin{align*}
    (1-r&)(1+r+r^2+\dots+r^n+r^{n+1}) \\
    &= (1-r)(1+r+r^2+\dots+r^{n}) + (1-r)r^{n+1} \\
    &= 1 - r^{n+1} + (1-r)r^{n+1} \\
    &= 1 - r^{n+1} + r^{n+1} - r\cdot r^{n+1} \\
    &= 1 - r\cdot r^{n+1} \\
    &= 1 - r^{(n+1)+1} \,.
\end{align*}
We've shown that $P_n$ implies $P_{n+1}$. By induction we conclude that
$P_n$ holds for all natural $n$.
\qedsymbol

\end{document}