\documentclass{article}
\usepackage{amsmath} 
\usepackage{amsfonts}
\usepackage{amssymb}
\usepackage[parfill]{parskip}
\usepackage{esvect}
\usepackage[thinc]{esdiff}
\usepackage{mathtools}
\usepackage{physics}
\usepackage{amsthm}
\usepackage{xfrac}
\usepackage{enumitem}

\DeclareMathOperator{\lub}{lub}
\DeclareMathOperator{\gub}{gub}

\begin{document}

\title{MATH 411: Week 3}
\author{Jacob Lockard}
\date{10 February 2026}

\maketitle

\section*{Lemma 1}

\begin{quote}
    \itshape
    Let $a_1, a_2, \dots, a_n \in \mathbb{R}$ be such that for all $i \in \{1, 2, \dots, n\}$
    we have $0 \leq a_i < M$ for some $M \in \mathbb{R}^+$.
    Then,
    \begin{align*}
        \prod_{i=1}^n a_i < M^n \,.
    \end{align*}
\end{quote}

We proceed by induction. For all $n \in \mathbb{N}$, let $P_n$ be the statement that if
$a_1, a_2, \dots, a_n \in \mathbb{R}$ and
$0 \leq a_i < M$ for all
$i \in \{1, 2, \dots, n\}$ for some $M \in \mathbb{R}^+$, then
$\prod_{i=1}^n a_i < M^n$.

If $a_1 \in \mathbb{R}$ and $0 \leq a_1 < M$ for some $M \in \mathbb{R}^+$, then
we have:
\begin{align*}
    \prod_{i=1}^1 a_1 = a_1 < M = M^1 \,.
\end{align*}
So $P_1$ holds.

Now assume that $P_n$ holds for some $n \in \mathbb{N}$.
Let $a_1, a_2, \dots, a_n, a_{n+1} \in \mathbb{R}$ with $0 \leq a_i < M$
for all $i \in \{1, 2, \dots, n, n+1\}$ for some $M \in \mathbb{R}^+$.
If $a_{n+1} = 0$, then since $M$ is positive,
\begin{align*}
    \prod_{i=1}^{n+1} a_i = (a_{n+1})\prod_{i=1}^n a_i = 0 < M^{n+1} \,.
\end{align*}
On the other hand, if $a_{n+1} > 0$, we have:
\begin{align*}
    \prod_{i=1}^n a_i &< M^n \\
    (a_{n+1})\prod_{i=1}^n a_i &< (a_{n+1})M^n \\
    \prod_{i=1}^{n+1} a_i &< (a_{n+1})M^n \,.
\end{align*}
Since $a_{n+1} < M$ and $M^n > 0$, it follows that
$(a_{n+1})M^n < M \cdot M^n = M^{n+1}$. We conclude that
\begin{align*}
    \prod_{i=1}^{n+1} a_i &< M^{n+1} \,.
\end{align*}
We've shown that $P_n$ implies $P_{n+1}$. So $P_n$ holds for all
natural $n$ by induction.

\qedsymbol

\section*{Lemma 2}

\begin{quote}
    \itshape
    Let $\{a_n\}$ be a monotonic sequence.
    If $\{a_n\}$ is increasing, then $\{a_n\} \to \infty$
    if and only if $\{a_n\}$ is not bounded above.
    Likewise, if $\{a_n\}$ is decreasing, then $\{a_n\} \to -\infty$
    if and only if $\{a_n\}$ is not bounded below.
\end{quote}

Clearly, if $\{a_n\} \to \infty$, then $\{a_n\}$ is not bounded above,
and if $\{a_n\} \to -\infty$, then $\{a_n\}$ is not bounded below.

Suppose $\{a_n\}$ is increasing and not bounded above.
Then, for any $M > 0$, there's a natural $N$ such that $a_N > M$.
Since $\{a_n\}$ is increasing, if $n > N$, then we still have $a_n > M$.
So $\{a_n\} \to \infty$.

Now suppose $\{a_n\}$ is decreasing and not bounded below.
Then, for any $M < 0$, there's a natural $N$ such that $a_N < M$.
Since $\{a_n\}$ is decreasing, if $n > N$, then we still have $a_n < M$.
So $\{a_n\} \to -\infty$.

\qedsymbol

\section*{Lemma 3}

\begin{quote}
    \itshape
    Let $\sum a_n$ be a series with sequence of partial sums $\{s_n\}$.
    If $a_n \geq 0$ for all natural $n$, then $\{s_n\} \to \infty$
    if and only if $\sum a_n$ diverges.
    Likewise, if $a_n \leq 0$ for all natural $n$, then $\{s_n\} \to -\infty$
    if and only if $\sum a_n$ diverges.
\end{quote}

If $a_n \geq 0$ for all natural $n$, then $\{s_n\}$ is clearly monotonically
increasing. The proof of Problem 5.23 ensures that $\{s_n\}$ is not bounded above
if and only if $\sum a_n$ diverges.

If $a_n \leq 0$ for all natural $n$, then $\{s_n\}$ is clearly monotonically
decreasing. The proof of Problem 5.23 ensures that $\{s_n\}$ is not bounded below
if and only if $\sum a_n$ diverges.

Lemma 2 finishes the proof.

\qedsymbol

\section*{Problem 7.21}

\begin{quote}
    \itshape
    For the series $\sum a_n$, suppose that there exists a real number
    $\beta$ with $0 < \beta < 1$ and an $N \in \mathbb{N}$ such that for all
    $n \geq N$ we have $\sqrt[n]{|a_n|} \leq \beta$. Prove that
    $\sum a_n$ converges absolutely.
\end{quote}

Let $\sum_{n=1}^\infty a_n$ be a series, let $\beta \in (0, 1)$, and let $N \in \mathbb{N}$,
with $\sqrt[n]{|a_n|} \leq \beta$ for all $n \geq N$. We'll show that
$\sum_{n=1}^\infty a_n$ converges absolutely.

For all $n \in \mathbb{N}$, since $|a_n|$ is nonnegative, we notice that
\begin{align*}
    \sqrt[n]{|a_n|} &\leq \beta \\
    |a_n| &\leq \beta^n \,.
\end{align*}
Since $\beta \in (0,1)$, Problem 7.11 ensures the series $\sum_{n=1}^\infty \beta^n$
converges. So Problem 7.7 ensures $\sum_{n=1}^\infty |a_n|$ converges,
since for all $n \in \mathbb{N}$ we certainly have
$\bigl| |a_n| \bigr| = |a_n| \leq \beta^n$. So $\sum_{n=1}^\infty a_n$
converges absolutely.

\qedsymbol

\section*{Problem 7.22}

\begin{quote}
    \itshape
    For the series $\sum a_n$, suppose that there exists a real number
    $\beta$ with $0 < \beta < 1$ and an $N \in \mathbb{N}$ such that for
    all $n \geq N$ we have $\Bigl| \frac{a_{n+1}}{a_n} \Bigr| < \beta$.

    \begin{enumerate}[label=(\alph*)]
        \item Prove that for all $n \geq N$ we have
        \begin{align*}
            \Biggl| \frac{a_n}{a_N} \Biggr| \, \beta^N < \beta^n \,.
        \end{align*}
        \item Use the result above to prove that $\sum a_n$ converges
        absolutely.
    \end{enumerate}
\end{quote}

Let $\sum_{n=1}^\infty a_n$ be a series, let $\beta \in (0,1)$, and let
$N \in \mathbb{N}$, with $\Bigl| \frac{a_{n+1}}{a_n} \Bigr| < \beta$
for all $n \geq N$.

\textbf{(a)} Let $n \in \mathbb{N}$ with $n \geq N$. We'll show that
\begin{align*}
    \Biggl| \frac{a_n}{a_N} \Biggr| \, \beta^N < \beta^n \,.
\end{align*}
We have:
\begin{align*}
    \Biggl| \frac{a_n}{a_N} \Biggr|
    = \Biggl| \prod_{i=N}^{n-1} \frac{a_{i+1}}{a_i} \Biggl|
    = \prod_{i=N}^{n-1} \Bigl| \frac{a_{i+1}}{a_i} \Bigr|
    < \prod_{i=N}^{n-1} \beta = \beta^{n-N}
    \,.
\end{align*}
where the first statement can be easily proven by induction,
the second uses a well known fact about the absolute value of a product, and
the third is justified by Lemma 1 above,
since for each $i$ we have $0 \leq \Bigl| \frac{a_{i+1}}{a_i} \Bigr| < \beta$, and $\beta > 0$.

Since $\beta^N > 0$, we conclude that
\begin{align*}
    \Biggl| \frac{a_n}{a_N} \Biggr| \, \beta^N < \beta^{n-N} \beta^{N}
    = \beta^n \,.
\end{align*}

\textbf{(b)} We will show that $\sum_{n=1}^\infty a_n$ converges absolutely.

We observe that by part (a), for all $n \geq N$, since $\beta^N > 0$,
\begin{align*}
    \sqrt[n]{\Bigl| \frac{a_n}{a_N}\beta^N \Bigr|} < \beta \,.
\end{align*}
So Problem 7.21 ensures the series $\sum_{n=1}^\infty \Bigl| \frac{a_n}{a_N} \beta^N \Bigr|$
converges. A simple consequence of Problem 7.20 is that
if a series converges, the series formed by multiplying each of its term by 
a constant also converges. Hence, since $\beta > 0$, we can say that
\begin{align*}
    \sum_{n=1}^\infty \Bigl| \frac{a^N}{\beta^N} \Bigr| \Bigl| \frac{a_n}{a_N} \beta^N \Bigr| = \sum_{n=1}^\infty |a_n|
\end{align*}
converges. So $\sum_{n=1}^\infty a_n$ converges absolutely.

\qedsymbol

\section*{Problem 7.23}

\begin{quote}
    \itshape
    Use the comparison test to prove that $\sum a_n$ converges absolutely
    if and only if both $\sum a_n^+$ and $\sum a_n^-$ converge absolutely.
\end{quote}

Let $\sum a_n$ be a series. We'll show that $\sum a_n$ converges
absolutely if and only if both $\sum a_n^+$ and $\sum a_n^-$ converge
absolutely.

Assume $\sum a_n$ converges absolutely. We observe that for any natural $n$,
\begin{align*}
    |a_n^+| \leq |a_n| \,,
\end{align*}
because if $a_n \geq 0$, then $|a_n^+| = |a_n|$, and if $a_n < 0$,
then $|a_n| > 0 = |0| = |a_n^+|$. Likewise, we have
\begin{align*}
    |a_n^-| \leq |a_n| \,,
\end{align*}
because if $a_n \geq 0$, then $|a_n| \geq 0 = |0| = |a_n^-|$, and if
$a_n < 0$, then $|a_n^-| = |a_n|$. The comparison test
ensures that both $\sum a_n^+$ and $\sum a_n^-$ converge absolutely.

Now assume $\sum a_n^+$ and $\sum a_n^-$ converge absolutely.
We note that, by construction, for any natural $n$,
\begin{align*}
    |a_n| = |a_n^+ + a_n^-| \leq |a_n^+| + |a_n^-| \,.
\end{align*}
Since $\sum a_n^+$ and $\sum a_n^-$ converge absolutely,
$\sum\bigl(|a_n^+| + |a_n^-|\bigr)$ converges.
The comparison test ensures $\sum a_n$ converges absolutely.

\qedsymbol

\section*{Problem 7.24}

\begin{quote}
    \itshape
    Prove that if $\sum a_n$ converges conditionally then $\sum a_n^+$
    must go off to $\infty$ and $\sum a_n^-$ must go off to $-\infty$.
\end{quote}

Let $\sum a_n$ be a series that converges conditionally to $L$. We'll show that
the sequence of partial sums $\{s_n\}$ of $\sum a_n^+$ converges to $\infty$ and
the sequence of partial sums $\{t_n\}$ of $\sum a_n^-$ converges to $-\infty$.

We first note that, for any natural $n$,
\begin{align*}
    \sum_{k=1}^n a_k = \sum_{k=1}^n (a_k^+ + a_k^-)
    = \sum_{k=1}^n a_k^+ + \sum_{k=1}^n a_k^- = s_n + t_n \,.
\end{align*}

Let $\epsilon > 0$. Then there's an $N_1$ such that if $n \geq N_1$, then
$|s_n + t_n - L| < \epsilon$.
Let $M < 0$, and suppose $\{s_n\}$ is not bounded above. Then there's an $N_2$ such that
if $n \geq N_2$, then $s_n > -M + \epsilon + L$. Let $n = \max\{N_1, N_2\}$.
Then we have
\begin{align*}
    s_n &> -M + \epsilon + L \\
    s_n - L +M &> \epsilon \,. \\
    s_n - L + M &> s_n + t_n - L\\
    M &> t_n \,.
\end{align*}
So $\{t_n\}$ is not bounded below.
On the other hand, let $M > 0$, and suppose $\{t_n\}$ is not bounded below.
Then there's an $N_2$ such that if $n \geq N_2$, then $t_s < -M + \epsilon + L$.
Let $n = \max{N_1, N_2}$. Then we have
\begin{align*}
    t_n &< -M + \epsilon + L \\
    t_n - L M &< \epsilon \,. \\
    t_n - L + M &< s_n + t_n - L\\
    M &< s_n \,.
\end{align*}
So $\{s_n\}$ is not bounded above.

Now we observe that $\sum a_n^+$ and $\sum a_n^-$ each converge
if and only if they converge absolutely. The case of $\sum a_n^+$
is trivial, since $|a_n^+| = a_n^+$ for all $n$.
In the case of $\sum a_n^-$, we note that $|a_n^-| = -a_n^-$ for all $n$.
Problem 7.20 ensures that if $\sum a_n^-$ converges, then so does
$\sum -a_n^- = \sum |a_n^-|$.
Problem 7.23 ensures that at least one of $\sum a_n^+$ and $\sum a_n^-$
does not converge absolutely. So one of them must diverge.

If $\sum a_n^+$ diverges, then Lemma 3 ensures $\{s_n\}$ converges
to $\infty$, which as we just noted implies $\{t_n\}$ is
not bounded below, which by Lemma 2 means $\{t_n\}$ converges to $-\infty$.
Likewise, if $\sum a_n^-$ diverges, then Lemma 3 ensures $\{t_n\}$
converges to $-\infty$, which as we just noted implies $\{s_n\}$ is not
bounded above, which by Lemma 2 means $\{s_n\}$ converges
to $\infty$. In either case, $\{s_n\} \to \infty$ and $\{t_n\} \to -\infty$.

\qedsymbol

\section*{Problem 7.25}

\begin{quote}
    \itshape
    Let $\sum a_n$ be a conditionally convergent series.

    \begin{enumerate}[label=(\alph*)]
        \item Prove that for any $n \in \mathbb{N}$ and any $M > 0$,
        there exists a $q > n$ such that
        \begin{align*}
            \sum_{k=n}^q a_n^+ > M \,.
        \end{align*}
        \item Prove that for any $M > 0$ and $\epsilon > 0$, that there
        exists an $N \in \mathbb{N}$ such that for any $n \geq N$ there
        exists a $q > n$ with
        \begin{align*}
            M < \sum_{k=n}^q a_n^+ < M + \epsilon \,.
        \end{align*}
    \end{enumerate}
\end{quote}

Let $\sum a_n$ be a conditionally convergent series.
Let $s_n$ denote the $n$-th partial sum of
$\sum a_n^+$.

\textbf{(a)}
Let $n \in \mathbb{N}$ and $M > 0$. We will show there
exists a $q > n$ such that
\begin{align*}
    s_q - s_{n-1} = \sum_{k=n}^q a_k^+ > M \,.
\end{align*}

Problem 7.24 ensures that the sequence of partial sums of $\sum a_n^+$
converges to $\infty$. So there is a $q \in \mathbb{N}$ such that
$s_q > M + s_{n-1}$. We have:
\begin{align*}
    s_q - s_{n-1} > M + s_{n-1} - s_{n-1} = M \,.
\end{align*}

\textbf{(b)} Let $M > 0$ and $\epsilon > 0$. We will show that there exists
an $N \in \mathbb{N}$ such that for any $n \geq N$ there exists a $q > n$
with
\begin{align*}
    M < \sum_{k=n}^q a_n^+ < M + \epsilon \,.
\end{align*}

Let $\{p_{n_k}\}$ be the subsequence of $\{a_n\}$ whose terms are positive.
Since $\sum a_n$ converges, Problem 7.5 and Problem 5.15 ensure that
$\{p_{n_k}\} \to 0$. We note that every term in $\{a_n^+\}$ is either
a term of $\{p_{n_k}\}$ or is equal to $0$. Since every neighborhood
around $0$ contains all but finitely many terms of $\{p_{n_k}\}$
and certainly contains $0$, we conclude that $\{a_n^+\} \to 0$.

Let $n \in \mathbb{N}$ be such that if $p \geq n$, then we have
\begin{align*}
    a_{p}^+ = |a_{p}^+ - 0| < \min \{\epsilon, M \} \,,
\end{align*}
which exists because $\epsilon$ and $M$ are positive.
Let $Q > n$ be such that
\begin{align*}
    \sum_{k=n}^{Q} a_k^+ > M \,,
\end{align*}
as guaranteed by part (a). Let $m$ be the largest element
of $\{n, n+1, \dots, Q\}$ that satisfies
\begin{align*}
    \sum_{k=n}^m a_k^+ \leq M \,.
\end{align*}
$m$ exists, since $a_n^+ < M$. We know that
\begin{align*}
    \sum_{k=n}^{m+1} a_k^+ > M \,,
\end{align*}
since $m \neq Q$ and thus $m+1 \in \{n, n+1, \dots, Q\}$, but $m + 1 > m$.
We also have:
\begin{align*}
    \sum_{k=n}^m a_k^+ &\leq M \\
    \sum_{k=n}^m a_k^+ + a_{m+1}^+ &\leq M + a_{m+1}^+ \\
    \sum_{k=n}^{m+1} a_k^+ &\leq M + a_{m+1}^+ \,.
\end{align*}
Further, $a_{m+1}^+ < \epsilon$, since $m + 1 > m \geq n$. So we have:
\begin{align*}
    \sum_{k=n}^{m+1} a_k^+ &\leq M + a_{m+1}^+ < M + \epsilon\,.
\end{align*}
So $q = m+1$ has the desired properties.

\qedsymbol

\section*{Problem 7.26}

\begin{quote}
    \itshape
    Prove that for any $\alpha, \beta \in \mathbb{R}$ with $\alpha < \beta$,
    if $\sum a_n$ converges conditionally, there exists a rearrangement
    of the series $\sum a'_n$ with partial sums $s'_n$ such that
    \begin{align*}
        \liminf_{n\to\infty} s'_n = \alpha, \; \text{and} \;
        \limsup_{n\to\infty} s'_n = \beta \,.
    \end{align*}
\end{quote}

We will not prove this formally, but will instead informally construct
such a rearrangment and justify why it has the desired properties.

We construct our sequence $\{a'_n\}$ by alternating between two steps:
\begin{enumerate}
    \item taking enough terms of $\{a_n^+\}$ so that their sum is sufficiently
    large in magnitude to make the whole sum greater
    than $\beta$ but as small as possible; and
    \item taking enough terms of $\{a_n^-\}$ so that their sum is sufficiently
    large in magnitude to make the whole sum less than $\alpha$ but as
    large as possible.
\end{enumerate}
In this process, we skip terms from $\{a_n^+\}$ and $\{a_n^-\}$ where
$a_n^+ \neq a_n$ or $a_n^- \neq a_n$. Since this can only occur when
$a_n = 0$, it does not change the incremental sums.
Since $\sum a_n^+$ goes off to $\infty$ and $\sum a_n^-$ goes off to $-\infty$, we
are assured that we can always take enough terms of each to produce the
stated behavior.

Every term of $\{a_n\}$ is either in $\{a_n^+\}$ or in $\{a_n^-\}$.
Since any term of $\{a_n^+\}$ and $\{a_n^-\}$ that is in $\{a_n\}$
is eventually represented, we know that every term of $\{a_n\}$ is represented.
Further, no term is repeated, since neither $\{a_n^+\}$ nor $\{a_n^-\}$
repeat terms from $\{a_n\}$, and neither one contains terms
of $\{a_n\}$ that the other contains.
So we are justified in claiming that $\{a_n'\}$ is a rearrangment of $\{a_n\}$.

There are an infinite number of terms of $\{s_n'\}$ greater than $\beta$---namely,
every partial sum obtained by summing up to the end of the
terms taken in any instance of step (1) above. Likewise, there
are an infinite number of terms of $\{s_n'\}$ less than $\alpha$.
Moreover, these are the only terms of $\{s_n'\}$ above $\beta$ and below $\alpha$,
since within a step the sum is monotonic, we made each post-step sum as small in magnitude as possible,
and after each step, the terms of $\{a_n'\}$ change sign.

We've shown in previous problems that since $\sum a_n$ converges,
$\{a_n^+\} \to 0$ and $\{a_n^-\} \to 0$. Let $\epsilon > 0$.
So there's an $N$ such that if $n \geq N$, we have
$a_n^+ < \epsilon$ and $a_n^- > -\epsilon$.
Consider a term $s_n'$ of $\{s_n'\}$ with $n \geq N$ such that $s_n' > \beta$.
We know that $s_{n-1}' \leq \beta$, since $s_n'$ is as small as possible. So we have:
\begin{gather*}
    s_n' - \beta \leq s_n' - s_{n-1}' = a_n^+ < \epsilon \\
    s_n' < \beta + \epsilon \,.
\end{gather*}
Likewike, for a term $s_n' < \alpha$, we have:
\begin{gather*}
    s_n' - \alpha \geq s_n' - s_{n-1}' = a_n^- > -\epsilon \\
    s_n' > \alpha -\epsilon \,.
\end{gather*}
We've shown that for any given $\epsilon > 0$, there are an infinite number of
terms of $\{s_n'\}$ in the $\epsilon$-neighborhood around $\alpha$ and $\beta$---in particular,
all the partial sums after a certain point obtained by summing up to the end of the terms taken in any
step above.
We conclude that $\alpha$ and $\beta$ are accumulation points for $\{s_n'\}$.

Further, since no other terms---besides the ones that we've just shown
converge to $\alpha$ and $\beta$---are above $\beta$ or below $\alpha$,
we know that $\beta$ is the biggest accumulation point
and $\alpha$ is the smallest accumulation point.

\qedsymbol

\section*{Problem 7.28}

\begin{quote}
    \itshape
    Suppose that $\sum a_n$ converges and $a_n \geq 0$ for all $n$. Prove 
    that $\sum \frac{\sqrt{a_n}}{n}$ converges as well.
\end{quote}

Suppose that $\sum a_n$ converges and $a_n \geq 0$ for all $n$. We'll show
that $\sum \frac{\sqrt{a_n}}{n}$ converges as well.
The Scharz inequality ensures that for any $N$,
\begin{align*}
    \Biggl(\sum_{n=1}^N \sqrt{a_n} \cdot \frac{1}{n} \Biggr)^2 \leq 
    \Biggl(\sum_{n=1}^N a_n\Biggr)\Biggl( \sum_{n=1}^N \frac{1}{n^2} \Biggr) \,.
\end{align*}
$\sum a_n$ converges by assumption and $\sum \frac{1}{n^2}$ converges since it's a $p$-series
with $p > 1$. Problem 5.13 then ensures that
\begin{align*}
    \Biggl\{ \Biggl(\sum_{n=1}^N a_n\Biggr)\Biggl( \sum_{n=1}^N \frac{1}{n^2} \Biggr) \Biggr\}_{n=1}^\infty
\end{align*}
converges. We note that, for all $N$,
\begin{align*}
    \sum_{n=1}^N \sqrt{a_n} \cdot \frac{1}{n} \geq 0 \,,
\end{align*}
since $a_n \geq 0$. Thus, the squeeze theorem ensures
\begin{align*}
    \Biggl\{\Biggl(\sum_{n=1}^N \sqrt{a_n} \cdot \frac{1}{n} \Biggr)^2\Biggr\}_{n=1}^\infty
\end{align*}
converges.
???

\qedsymbol

\section*{Problem 7.29}

Let $\sum a_nx^n$ be a power series for some $x$. Let
\begin{align*}
    \alpha = \limsup_{n\to\infty} \sqrt[n]{|a_n|}
\end{align*}
and
\begin{equation*}
    R = \begin{cases}
        0 &\quad \alpha = \infty \\
        \frac{1}{\alpha} &\quad 0 < \alpha < \infty \\
        \infty &\quad \alpha = 0
    \end{cases} \,.
\end{equation*}
We will show that $\sum a_nx^n$ converges absolutely for all $|x| < R$
and diverges for all $|x| > R$.

Assume $\alpha = \infty$. Then $\{\sqrt[n]{|a_n|}\}$ is unbounded and
for any $M > 0$ there's an $N$ such that if $n \geq N$ then
\begin{align*}
    \sqrt[n]{|a_n|} &> M \\
    |a_n| &> M^n \,.
\end{align*}
So $\{|a_n|\}$ is unbounded and hence $\sum a_n$

\end{document}
