\documentclass{article}
\usepackage{amsmath} 
\usepackage{amsfonts}
\usepackage{amssymb}
\usepackage[parfill]{parskip}
\usepackage{esvect}
\usepackage[thinc]{esdiff}
\usepackage{mathtools}
\usepackage{physics}
\usepackage{amsthm}
\usepackage{xfrac}
\usepackage{enumitem}

\DeclareMathOperator{\lub}{lub}
\DeclareMathOperator{\gub}{gub}

\begin{document}

\title{MATH 411: Week 2}
\author{Jacob Lockard}
\date{5 February 2026}

\maketitle

\section*{Lemma}

\begin{quote}
    \itshape
    Let $\sum_{k=k_0}^\infty a_k$ be a series. Let $c$ be a nonzero real number.
    Then $\sum_{k=k_0}^\infty a_k$ converges if and only if $\sum_{k=k_0}^\infty c a_k$
    converges.
\end{quote}

Let $\{s_n\}$ and $\{t_n\}$ be the sequences of partial sums of $\sum a_k$
and $\sum ca_k$, respectively.
By Problem 5.11, if $\{s_n\}$ converges, then $\{cs_n\}_{n = k_0}^\infty$ converges.
On the other hand, if $\{cs_n\}_{n=k_0}^\infty$ converges, then since $c \neq 0$,
Problem 5.11 ensures that $\{\frac{1}{c} c s_n\} = \{s_n\}$ converges.
So we've shown that $\{s_n\}$ converges if and only if $\{cs_n\}$
converges.

We observe that, for all $n \geq k_0$,
\begin{align*}
    c s_n = c \sum_{k=k_0}^n a_k = \sum_{k=k_0}^n ca_k = t_n \,.
\end{align*}
We conclude that $\{s_n\}$ converges if and only if $\{t_n\}$ converges, and hence
that $\sum a_k$ converges if and only if $\sum ca_k$
converges.
\qedsymbol

\section*{Problem 7.15}

\begin{quote}
    \itshape
    Use Theorem 7.1 to prove that the series $\sum_{n=1}^\infty \frac{1}{n^p}$
    converges if and only if $p > 1$.
\end{quote}

Let $p \in \mathbb{R}$. We will show that the series $\sum_{n=1}^\infty \frac{1}{n^p}$
converges if and only if $p > 1$. Problem 7.1 ensures that
$\sum_{n=1}^\infty \frac{1}{n^p}$ converges if and only if
\begin{align*}
    \sum_{n = 0}^\infty 2^n \frac{1}{(2^n)^p}
    = \sum_{n = 0}^\infty \frac{1}{(2^n)^{p-1}}
    = \sum_{n = 0}^\infty \frac{1}{(2^{p-1})^{n}}
    = \sum_{n = 0}^\infty \Bigl(\frac{1}{2^{p-1}}\Bigr)^n
\end{align*}
converges. By Problem 7.11, this series converges if and only if
\begin{gather*}
    \Bigl| \frac{1}{2^{p-1}} \Bigr| < 1 \\
    \bigl|2^{p-1}\bigr| > 1 \\
    p-1 > 0 \\
    p > 1 \,.
\end{gather*}

\qedsymbol

\section*{Problem 7.16}

\begin{quote}
    \itshape
    Use Theorem 7.1 to prove that the series
    $\sum_{n=1}^\infty \frac{1}{n (\ln n)^p}$ converges if and only if
    $p > 1$.
\end{quote}

Let $p \in \mathbb{R}$. We will show that the series
$\sum_{n=1}^\infty \frac{1}{n (\ln n)^p}$ converges if and only if
$p > 1$. Problem 7.1 ensures that $\sum_{n=1}^\infty \frac{1}{n^p}$ converges
if and only if
\begin{align*}
    \sum_{n=0}^\infty 2^n \frac{1}{2^n (\ln 2^n)^p}
    &= \sum_{n=0}^\infty \frac{1}{(\ln 2^n)^p} \\
    &= \sum_{n=0}^\infty \frac{1}{(n \ln 2)^p} \\
    &= \sum_{n=0}^\infty \frac{1}{n^p (\ln 2)^p} \\
    &= \sum_{n=0}^\infty \Biggl( \frac{1}{(\ln 2)^p} \cdot \frac{1}{n^p} \Biggr)
\end{align*}
converges. By our lemma above and Problem 7.15,
this series converges if and only if $p > 1$, since
$\frac{1}{(\ln 2)^p} \neq 0$.

\qedsymbol

\section*{Problem 7.17}

\begin{quote}
    \itshape
    For a series $\sum a_n$ suppose that there exists some $M > 0$
    such that for each partial sum $s_n$ we have $|s_n| < M$. Prove that
    for any $0 < p < q$ we have
    \begin{align*}
        \Biggl| \sum_{k=p}^q a_k \Biggr| < 2M \,.
    \end{align*}
\end{quote}

Let $M > 0$, and let $\sum a_n$ be a series such that for each partial sum
$s_n$ we have $|s_n| < M$. Let $0 < p < q$. We'll show that
\begin{align*}
    \Biggl| \sum_{k=p}^q a_k \Biggr| < 2M \,.
\end{align*}
We have:
\begin{align*}
    \Biggl| \sum_{k=p}^q a_k \Biggr| &= \Biggl| \sum_{k=1}^q a_k - \sum_{k=1}^p a_k \Biggr| \\
    &= |s_q + (- s_p)| \\
    &< |s_q| + |-s_p| \\
    &= |s_q| + |s_p| \\
    &< M + M \\
    &< 2M \,.
\end{align*}
\qedsymbol


\section*{Problem 7.19}

\begin{quote}
    \itshape
    Suppose that $\{c_n\}$ is a monotonically decreasing sequence such
    that $\lim_{n\to\infty} c_n = 0$. Use Theorem 7.2 to prove that
    the series $\sum (-1)^n c_n$ converges.
\end{quote}

Suppose that $\{c_n\}$ is a monotonically decreasing sequence such
that $\lim_{n\to\infty} c_n = 0$. We'll show that
the series $\sum (-1)^n c_n$ converges.

If $n$ is even, then $\sum_{k=1}^n (-1)^k = 0$, and when
$n$ is odd, $\sum_{k=1}^n (-1)^k = -1$. In either case,
$\Bigl|\sum_{k=1}^n (-1)^k\Bigr| < 2$, so the sequence of partial sums of
$\{(-1)^n\}$ is bounded.
Then since $\{c_n\}$ is monotically decreasing with $\{c_n\} \to 0$,
Theorem 7.2 ensures $\sum (-1)^n c_n$ converges.

\qedsymbol

\end{document}